\section*{Sibilla}
Sibilla \`e un ambiente di simulazione specificatamente pensato per supportare la simulazione di sistemi basati su un elevato numero di agenti che interagiscono.
%
Il framework non si basa su un particolare formalismo, ma fornisce gli strumenti per poter personalizzare sia il processo di simulazione che la tipologia di modello utilizzato per descrivere i sistemi.
%
Questo \`e stato grazie all'uso degli opportuni pattern di sviluppo che semplificano l'integrazione di nuovi componenti.

Attualmente in Sibilla sono integrati modelli per la descrizione dei sistemi per mezzo di \emph{Process Description Languages} (PDL). Questa tipologia di formalismi consentono di rappresentare il sistema in termini di una serie di \emph{processi} o \emph{agenti} che interagiscono tra loro allo scopo di raggiungere un particolare obiettivo.
%
Tali formalismi fondano le proprie basi teoriche nelle \emph{Algebre di Processo}, strumenti matematici specificatamente introdotti per descrivere il \emph{comportamento} e la \emph{comunicazione} tra i processi.

Il framework Sibilla, inoltre, consente di poter effettuare la simulazione utilizzando un approccio multi-threading e concorrente. 
%
Questo permette di poter sfruttare al meglio le capacit\`a computazionali dei moderni processori che, grazie alla loro struttura basata su \emph{multicore}, permettono l'esecuzione concorrente di un numero di flussi computazionali (o \emph{thread}).

Il lavoro del progetto si è concentrato nello sviluppo e raffinamento delle classi orientate a rendere la simulazione del framework \emph{Sibilla} distribuita in rete.
%
Da questo lavoro è nata la libreria \texttt{quasylab.sibilla.core.network}, ideata per affiancare e sfruttare le classi già presenti nella libreria originale \texttt{quasylab.sibilla.core.simulator}.