\section{Documentazione del codice}
Il codice della libreria sviluppata è stato documentato tramite \textbf{JavaDoc}. A partire dai commenti JavaDoc scritti, la generazione dell'intera documentazione del progetto è stata affidata a \href{https://www.doxygen.nl/index.html}{\textbf{Doxygen}}, tool \textit{standard de facto} per la generazione della documentazione per progetti scritti in molteplici linguaggi di programmazione, tra cui Java.
All'interno della cartella root del \href{https://github.com/quasylab/sibilla/tree/working}{repository GitHub della libreria Sibilla} sono presenti due file che permettono una generazione agevole della documentazione tramite il tool citato. Il primo, \texttt{doxygenConfig}, è il file di configurazione sul quale si poggia Doxygen mentre il secondo, \texttt{documentation.sh}, è uno script per la bash Unix che avvia Doxygen generando documentazione in formato \texttt{.html} e \texttt{.pdf} all'interno della cartella \texttt{/docs}.

Lo script in bash richiede che Doxygen sia installato nel sistema assieme alle sue dipendenze. Un'alternativa alla generazione della documentazione del codice, indirizzata agli utenti Windows e MacOS, risiede in \href{https://www.doxygen.nl/manual/doxywizard_usage.html}{\textbf{DoxyWizard}}, frontend che permette di eseguire il tool Doxygen sfruttando il file di configurazione incluso nel progetto e di ottenere il medesimo tipo di documentazione che si otterrebbe tramite la bash Unix.

Per i fini di presentazione della libreria sviluppata, all'interno del \href{https://github.com/quasylab/sibilla/tree/working}{repository GitHub di riferimento} è possibile trovare una cartella \texttt{/docs} all'interno della quale è già presente tutta la documentazione generata a partire dai JavaDoc, sia in formato \texttt{.html} che \texttt{.pdf}.