
\section{Descrizione dell'infrastruttura}
L'architettura alla base delle comunicazioni tra i vari nodi della libreria è di natura \textbf{master/slave}.
Più specificatamente, le simulazioni da eseguire sono sottomesse da parte di un \textbf{client} che si connette ad un \textbf{server master} disponibile pubblicamente in rete, da cui vengono provengono anche i risultati delle simulazioni. 
All'interno della rete locale al master sono quindi presenti i \textbf{server slave} che rappresentano le unità di elaborazione delle simulazioni. Questi server non sono disponibili pubblicamente in rete e interagiscono con il master per poter ricevere nuove simulazioni da eseguire e per poter restituire i risultati di tali simulazioni.

\subsection{Client}
La logica di funzionamento di un client è contenuta interamente nella classe \texttt{ClientSimulationEnvironment}, le cui istanze devono essere incluse in tutte le classi di avvio di un client.
Nella definizione della classe di avvio di un client server è necessario includere l'istanziazione di un oggetto della classe \texttt{ModelDefinition}, rappresentante il modello della simulazione che verrà sottomesso per essere elaborato, e parametri relativi alla simulazione quali il numero delle repliche e la deadline [?].

Alla sua creazione, l'istanza di \texttt{ClientSimulationEnvironment} cercherà di contattare tramite la rete un server master utilizzando i parametri definiti all'avvio, quali porta, indirizzo IP e tipo di comunicazione basata su TCP. 
Durante questa fase vengono trasmessi al server master i byte contenuti nel file compilato .class relativo alla classe che implementa \texttt{ModelDefinition}, istanziata all'avvio del client.
Il caricamento di queste informazioni nel server master risulta fondamentale per poter gestire correttamente i dati e i parametri relativi alla simulazione che sono trasmessi dal client successivamente alla prima fase.

L'invio di questi dati coincide con la sottomissione effettiva della simulazione al server master. 
Tutte le comunicazioni successive a questa fase riguardano la ricezione dei risultati da parte del server master e la chiusura della comunicazione sia lato client che lato server master.

\subsection{Server Master}

Sottosezione del master

\subsection{Server Slave}

