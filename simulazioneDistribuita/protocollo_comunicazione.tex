
\section{Protocollo di comunicazione}

I tre componenti dell'infrastruttura comunicano tra di loro tramite l'invio di pacchetti sulla rete, utilizzando un protocollo di comunicazione personalizzato. I messaggi sono di due possibili tipi: comandi o dati. I comandi sono dei messaggi che danno indicazioni agli altri componenti riguardo i dati che verranno inviati e riguardo alle particolari azioni da eseguire, mentre i dati sono le informazioni che vengono utilizzate per eseguire le azioni richieste dai comandi. In generale entrambi i tipi di messaggi sono composti da degli oggetti Java serializzati ed inviati sulla rete.

\subsection{Comandi scambiati}

\subsubsection{Client}

\begin{table}[H]
    \begin{tabularx}{\linewidth}{ l X }
       \texttt{INIT}             & Indica l'inizio di una connessione con un master server, è seguito dall'invio del nome della classe \texttt{ModelDefinition} da simulare e dai corrispondenti class bytes \\
       \texttt{DATA}             & Indica l'invio dei dati della simulazione da eseguire, è seguito dall'invio del \texttt{SimulationDataSet} da simulare \\
       \texttt{PING}             & Invia una ping request ad un server \\
       \texttt{CLOSE\_CONNECTION} & Indica la chiusura della connessione con l'host remoto
    \end{tabularx}
\end{table}

\subsubsection{Master}

\begin{table}[H]
    \begin{tabularx}{\linewidth}{ l X }
        \texttt{INIT} & \\
        \texttt{PING} & \\
        \texttt{TASK} & \\
        \texttt{RESULTS} & \\
        \texttt{PONG} & \\
        \texttt{INIT\_RESPONSE} & \\
        \texttt{DATA\_RESPONSE} & \\
        \texttt{CLOSE\_CONNECTION} & \\
    \end{tabularx}
\end{table}

\subsubsection{Slave}

\begin{table}[H]
    \begin{tabularx}{\linewidth}{ l X }
        \texttt{PONG} & \\
        \texttt{INIT\_RESPONSE} & \\
        \texttt{CLOSE\_CONNECTION} & \\
    \end{tabularx}
\end{table}

\subsection{Serializzazione e compressione}

\subsection{Network Manager}