\section{Sviluppi futuri della libreria}
Gli sviluppi futuri della libreria proposta muovono direttamente dai limiti e dalle problematiche riscontrate durante lo sviluppo e sono riassumibili in:

\begin{itemize}
    \item \emph{Sistema di serializzazione avanzato} che permetta di ridurre significativamente i tempi di serializzazione di risultati di simulazione che presentino dimensioni importanti in termini di memoria.
    \item \emph{Miglioramento della gestione dell'invio dei risultati} da parte di un server Slave verso un server Master e da parte di un server Master verso un Client per evitare problemi di saturazione della rete con conseguente perdita di pacchetti e ritardi nella comunicazione.
    \item \emph{Revisione e ottimizzazione del meccanismo di coordinamento e bilanciamento} del carico relativo ai task di simulazione inviati dal server Master al server Slave. Il meccanismo attualmente presente è da attribuire ad un lavoro precedente all'inizio dei lavori sulla libreria presentata.
    \item \emph{Ottimizzazione orientata a dispositivi dalle limitate risorse di calcolo}, come i RaspberryPi impiegati durante le fasi di sviluppo. Nello specifico si prospetta la possibilità di poter testare il funzionamento della libreria, con conseguenti studi sull'ottimizzazione, su un \emph{cluster di RaspberryPi} impiegati come server Slave.
    \item \emph{Ulteriore sviluppo dell'interfaccia di monitoraggio} sviluppata a partire dall'esempio di avvio del server Master.
\end{itemize}