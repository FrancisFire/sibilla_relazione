\section{Problemi e limiti della libreria}

Abbiamo notato come all'interno dell'infrastruttura il processo di serializzazione dei risultati di una simulazione eseguita da parte di un server Slave sia un fattore problematico, in quanto può impiegare una grande quantità di tempo. In particolare va notato che la serializzazione di un risultato non dipende direttamente dalla complessità di un modello o dalla quantità di tempo necessaria per calcolare un risultato, ma unicamente dalla grandezza in termini di memoria del risultato. Questo significa che durante la simulazione di modelli semplici la performance dell'infrastruttura possa risultare molto minore rispetto ad una simulazione su vari thread di un unico dispositivo, dove appunto non è richiesta serializzazione. Tale divario viene sempre più ammortizzato nei casi in cui la computazione di un risultato richieda più tempo di computazione, in quel caso il tempo di serializzazione, che non dipende dalla complessità del modello, potrebbe diventare trascurabile.

Un'altro limite dell'infrastruttura è la necessità di dover trasferire molti dati su rete. In particolare questo potrebbe provocare ulteriori problemi nel caso in cui alcuni messaggi vengano persi nella rete. Inoltre bisogna considerare che durante l'esecuzione di una simulazione da parte di un server Slave, ai tempi di computazione e di serializzazione vanno aggiunti i tempi di trasferimento su rete dei risultati. Anche questo tempo dipende unicamente dalla grandezza in termini di memoria dei risultati scambiati. Bisogna inoltre notare che i risultati vengono spediti al server Master attraverso la rete solo al termine di una simulazione da parte del server Master, questo genera dei picchi di carico molto alti in determinati istanti, mentre in altri istanti non abbiamo alcun pacchetto scambiato sulla rete.

Bisogna inoltre notare che l'intera infrastruttura è stata progettata per essere funzionante su un cluster di RaspberryPi da utilizzare come server Slave. Tuttavia a causa di limitazioni tecniche non è stato possibile testare tale infrastruttura su questi dispositivi in modo corretto. Non è stato quindi possibile risolvere determinati problemi che sono derivanti dalle specifiche hardware di un RaspberryPi.