\section{Dispositivi su cui è stata testata la libreria}
Il corretto funzionamento della libreria è stato comprovato dallo studio dei file di log derivati dall'esecuzione delle classi di avvio su un ampio range di dispositivi.
In tutti i dispositivi impiegati sono stati installati \textbf{Gradle} e un \textbf{Java JDK} in versione \textbf{11 o superiore}.
Nello specifico, durante lo sviluppo, l'interesse si è concentrato sull'hardware fornito dai dispositivi che sono stati utilizzati come server Master e come server Slave, fondamentale per poter comprendere la natura \textit{plug and play} della libreria e per poter studiare i vantaggi computazionali derivanti dall'utilizzo dell'architettura distribuita rispetto a quella locale presente nel framework originale.
Tra i dispositivi impiegati per lo sviluppo e il testing della libreria figurano:

\begin{itemize}
    \item Laptop, con CPU \textbf{Intel Core i7-7700HQ} (4 Core a 2,80 GHz con turbo massimo a 3,80 GHz e 8 thread), RAM da \textbf{16 GB} e OS \textbf{Ubuntu} eseguito tramite \textbf{Windows Subsystem for Linux} su base \textbf{Windows 10}.
    \item Laptop, con CPU \textbf{Intel Core i7-7700HQ} (4 Core a 2,80 GHz con turbo massimo a 3,80 GHz e 8 thread), RAM da \textbf{16 GB} e OS \textbf{Windows 10}.
    \item Desktop, con CPU \textbf{Intel Core i5-3570K} (4 core a 3,40 GHz con turbo massimo a 3,80 GHz e 4 thread), RAM da \textbf{8 GB} e OS \textbf{Linux Mint}.
    \item RaspberryPi 3, con CPU \textbf{Broadcom BCM2837} (4 core a 1,20 GHz), RAM da \textbf{1 GB} e OS \textbf{Raspbian Buster Lite}.
\end{itemize}
